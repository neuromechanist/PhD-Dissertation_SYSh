\documentclass[../thesis_seyed.tex]{subfiles}
% \graphicspath{{\subfix{img/}}}
\chapter{INTRODUCTION}
% chapter headings must be typed in all caps as any \MakeUppercase command does not transfer to the PDF file through hyperref.
Chapter and major headings should be typed in all caps.  Note that Chapter titles should be formatted and positioned exactly the same as frontmatter and other major headings. However, chapters with subtitles may be stacked, single-spaced, rather than appear on one line.
The Introduction presents an overview of the thesis or dissertation material to be discussed. For sample theses and dissertations, including sample Introductions from your discipline, visit the University Writing Center’s Graduate Gateway, located at http://www.uwc.ucf.edu. Please be aware that UWC links are for content samples only, not format samples.



\section{First-level Subheading}
First-level subheadings are centered, and occur in title case (upper/lower case letters). 
% Subheading formatting is set in the class file.  If you choose to use the alternative subheading formatting from the Thesis and Dissertation Manual or another style guideline, you will need to alter the subsection commands in the class style.

\subsection{Second-level Subheading}
Second-level Subheadings are usually centered in title case with no additional formatting. 

\subsubsection{Third-level Subheading}
Third-level subheadings are underlined and left-justified, still in title case. 

\paragraph{Fourth-level Subheading}
Fourth-level headings look like second-level headings, except that fourth-level headings are justified.

\subparagraph{Fifth-level Subheading}
The maximum number of subheadings you may use is five.  The fifth-level subheading is indented and underlined. 
