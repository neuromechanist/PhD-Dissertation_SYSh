\documentclass[../thesis_seyed.tex]{subfiles}
% \graphicspath{{\subfix{img/}}}
\begin{document}
\chapter{INTRODUCTION}
% chapter headings must be typed in all caps as any \MakeUppercase command does not transfer to the PDF file through hyperref.
\section{Aims and hypotheses}
Balance and walking issues are major problems with aging; >3 million older adults visit ER each year because of fall-related injuries, and >300,000 fall-related hip-fractures cost the US >\$50 billion. This burden would only dramatically increase as the U.S.'s aging population will double from 57 million in 2010 to 109 million in 2050 \cite{noauthor_undated-ql}. Seated locomotor exercise, such as cycling and recumbent stepping, share similar underlying neural correlates as walking \cite{Stoloff2007-da} but does not require maintaining balance. So, perturbed seated locomotor tasks might help the elderly be more prepared for perturbations during walking and reduce their fall risks. Recent studies have demonstrated significant improvement in walking and balance performance after unperturbed seated exercise for stroke and Parkison's disease patients \cite{Zhou2018-yy, Klarner2016-wh}. However, the motor responses and neural correlations of perturbations in seated locomotor tasks have not been studied adequately.

To determine the potential of the perturbed seated locomotor tasks with walking rehabilitation, we needed to quantify and compare biomechanical, muscular, and cortical responses during the perturbed seated task to the brain and body response to perturbation during walking. Importantly, previous studies have suggested error-based adaptation during perturbed walking. Subjects adapt their movement to reduce errors as they gain more experience with the perturbations and the adaptation would wash out soon after removing the perturbations \cite{Torres-Oviedo2011-tv}. Also, locomotor perturbations tend to increase the muscular activity and co-contraction, which would decrease with adapting to the perturbations \cite{Finley2013-lu, Acuna2019-lo}. On the cortical level, the anterior cingulate cortex located at the mid-prefrontal area of the brain monitors errors during locomotor perturbations \cite{Peterson2018-ht} and monitors walking during double support and toe-off \cite{Bradford2016-kp, Bulea2015-dv}. A recent study on the corticomuscular connectivity of perturbed balance-bean walking revealed a strong connection between the anterior cingulate cortex and the motor cortex's supplementary motor areas is present shortly after the perturbations \cite{Peterson2019-wz}.

The overarching aim of this research was to quantify the motor, muscular and cortical responses of young and older adults to locomotor perturbations during recumbent stepping, a seated arms and leg locomotor task. We aimed to contrast young and older adults' responses to understand the possible underlying correlations of aging on human locomotion during perturbed seated locomotion. We also aimed to quantify the possible differential responses to varied perturbation timings. We had a set of hypotheses for this research: 1) Subjects would initially increase their motor errors in response to perturbations. The errors would decrease as they adapt to the perturbations. 2) Brain dynamics, especially at the anterior cingulate cortex, would reflect and monitor the presence of the perturbations. 3) Brain activity would decrease as subjects gain more experience with the perturbations. 4) Both young and older adults would increase co-contraction of the agonist and antagonist muscles in response to perturbations. 5) Young adults would benefit from more cortical activity during the perturbations, while older adults would benefit from the perturbation with greater co-contraction than young adults. 6) young adults would have stronger corticomuscular connectivity than older adults in response to perturbations.

\section{Research oveview}
We followed the research aim in two main directions: 1) We determined the accuracy and reliability of the electrocortical source estimation with the current technological advancements. Specifically, we investigated the effects of precise digitization of 3D localizing the electroencephalography (EEG) electrodes on source estimation. 2) We tested young and older adults' cortical, muscular and biomechanical performance during perturbed recumbent stepping. Overall, 17 young adults and 11 older adults completed four perturbed stepping tasks. Each task included six minutes of perturbations during each stride, padded by two minutes of unperturbed stepping before and after the perturbations. We perturbed the stepping movement with 200ms of increased resistance at the extension-onset or mid-extension of the left or right leg. We only used one perturbation timing in each task.

The first two chapters of this dissertation are dedicated to quantifying EEG source estimation accuracy based on the reliability of the electrode locations. The first chapter deals with the variability of the different digitizing methods and how this variability can cause uncertainties in source estimation. The second chapter raises awareness about the importance of recording fiducial locations, i.e., the anatomical landmarks used to correlate the brain's electrode locations. We emphasized that failing to record the fiducial locations accurately can result in twice as large errors in source estimation as the original error.

We discussed the biomechanics and electrocortical correlates of perturbed recumbent stepping in young and older adults in the next three chapters. We quantified the motor and electrocortical responses to perturbations in young adults in chapter 3. We discussed that the brief perturbations during recumbent stepping would not create error-based adaptation, resulting in sustained motor modifications after removing the perturbations. We could also show that the perturbations elicit theta-band (3-8Hz) synchronization (i.e., phase-locked neuronal firing) in the anterior cingulate cortex and the supplementary motor areas and that the extension-onset perturbations resulted in stronger anterior cingulate theta-band synchronization than the mid-extension perturbations.

We compared the motor error and the muscular co-contraction of young and older adults in chapter 4. We found that older adults also do not follow the error-based learning paradigm to respond to the perturbations. Young adults used a wider range of their muscles than older adults to drive the stepper during perturbations. However, older adults could reduce their co-contraction in select muscles for each task to overcome the perturbations.

\bibliographystyle{ieeetr}
\bibliography{../refs}

% \printbibliography
\end{document}